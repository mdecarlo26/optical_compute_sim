\documentclass[conference]{IEEEtran}
\usepackage{graphicx}
\usepackage{amsmath, amssymb}
\usepackage{cite}
\usepackage{hyperref}
\usepackage{xcolor}

\title{\textbf{Simulation of Optical Computing for High-Speed AI and Signal Processing}}
\author{\IEEEauthorblockN{Santiago Sosa\& Marc DeCarlo}\\
\IEEEauthorblockA{Department of Electrical and Computer Engineering, Drexel University\\
Email: \{ss5427, mad534\}@drexel.edu}}
\begin{document}
\maketitle

\begin{abstract}
As modern electronic processors encounter escalating power and speed constraints, optical computing has garnered attention for its potential to enable ultra-fast, energy-efficient operations. In particular, the inherent parallelism of optical signals and their natural capacity to execute Fourier transforms make them well-suited for computationally intensive tasks such as convolution in deep neural networks and signal processing. This proposal investigates the feasibility of optical computing through simulation-based approaches, focusing on Fourier-optics-based matrix operations, wavelength-division multiplexing (WDM) for parallelism, and hybrid optical-electronic systems for AI acceleration. We outline a structured methodology using industry-standard tools, analyze anticipated performance benefits, and compare optical architectures against conventional GPU- and FPGA-based electronic systems. This work aims to establish a comprehensive simulation framework that integrates device-level modeling with AI workflows, providing a foundation for future photonic hardware development.
\end{abstract}

\begin{IEEEkeywords}
Optical Computing, Wavelength-Division Multiplexing, Fourier Optics, AI Acceleration, Signal Processing
\end{IEEEkeywords}

\section{Introduction}
Emerging applications in artificial intelligence and real-time signal processing demand ever-increasing computational throughput and energy efficiency. Conventional silicon-based processors (CPUs, GPUs, TPUs) have made remarkable progress in recent years; yet, they remain constrained by interconnect bottlenecks, dynamic power consumption, and thermal dissipation~\cite{jouppi2017datacenter}.

Optical computing offers an intriguing alternative by leveraging the parallel nature of light and the ability of optical systems to perform fast Fourier transforms. Early research demonstrated the feasibility of optical matrix multiplication and optical correlators~\cite{goodman2005introduction}, while more recent studies have explored integrated photonics for neural network acceleration~\cite{shen2017deep}. Key advantages of optical computing include:
\begin{itemize}
    \item \textbf{Parallelism}: Multiple wavelengths and spatial channels process data simultaneously.
    \item \textbf{Energy Efficiency}: Reduced resistive losses compared to electronic interconnects.
    \item \textbf{High Bandwidth}: Optical links can modulate and transmit data at terabit per second rates.
\end{itemize}

However, realizing these benefits in practical systems requires robust modeling of optical propagation, device-level considerations (losses, crosstalk, misalignment), and seamless integration with electronic hardware for tasks such as nonlinear activation. This proposal aims to address these challenges by developing a detailed simulation framework that captures both the physics of optical signal processing and the performance requirements of AI workloads.

\section{Objectives}
The primary goals of this research are:
\begin{enumerate}
    \item \textbf{Fourier Optics Convolution Simulation}: Implement and assess lens-based (or waveguide-based) Fourier transforms for matrix multiplications and convolutions relevant to deep learning.
    \item \textbf{WDM Parallelism Model}: Investigate how multiple optical channels operating at different wavelengths can execute parallel computations and evaluate potential crosstalk or interference.
    \item \textbf{Hybrid Optical-Electronic AI Pipelines}: Develop a framework where optical accelerators handle linear operations (e.g., convolutions), while electronic hardware executes nonlinear activations.
    \item \textbf{Comparative Benchmarks}: Compare optical simulations against GPU- and FPGA-based solutions in terms of speed, energy consumption, accuracy, and scalability.
    \item \textbf{Guidance for Hardware Prototypes}: Provide recommendations for future physical implementations, including key system parameters and design considerations.
\end{enumerate}

\section{Literature Review}
Optical computing research has evolved significantly, from early free-space optical correlators~\cite{goodman2005introduction} to on-chip photonic implementations for neural network layers~\cite{shen2017deep}. Recent work explores hybrid optical-electronic architectures, where components such as Mach-Zehnder interferometers or microring resonators serve as high-speed multipliers or linear transform blocks~\cite{hughes2018training}. Despite these advances, challenges remain, including limited precision due to quantization noise and fabrication tolerances, coupling losses in photonic waveguides, and practical limits on WDM channel counts. 

Moreover, existing demonstrations often focus on either:
\begin{itemize}
    \item \textbf{Device-Level Proof-of-Concepts}: Fabricating photonic circuits for small-scale matrix multiplication.
    \item \textbf{System-Level Designs}: Conceptual frameworks and high-level theoretical estimates of performance gains.
\end{itemize}
A unified simulation approach can help bridge these areas by incorporating physical modeling of optical components into system-level AI workloads.

\section{Methodology}
To evaluate the feasibility and performance of optical computing for AI acceleration, we will utilize a multi-phase simulation approach. Our methodology is divided into the following core components:

\subsection{Fourier Optics for High-Speed Convolutions}
\begin{itemize}
    \item \textbf{4f System Modeling}: Develop a MATLAB/Python-based simulation of a lens-based 4f system, which naturally implements the Fourier transform. Validate accuracy by comparing optical convolution outputs to standard digital FFT-based methods.
    \item \textbf{Device-Level Parameters}: Incorporate realistic optical effects such as misalignment, lens aberrations, and diffraction to gauge error margins and energy efficiency.
    \item \textbf{Benchmarking}: Compare speed and accuracy with GPU-based convolutions in PyTorch or TensorFlow. 
\end{itemize}

\subsection{Wavelength-Division Multiplexing (WDM) Parallelism}
\begin{itemize}
    \item \textbf{Channel Modeling}: Simulate multiple optical channels at different wavelengths, each performing partial computations. Evaluate potential crosstalk and channel spacing constraints.
    \item \textbf{Parallel Convolution}: Explore how WDM can be leveraged to process multiple convolution filters or image patches in parallel, increasing throughput.
    \item \textbf{Performance Comparison}: Measure effective throughput, energy consumption, and potential signal degradation relative to parallel GPU operations (CUDA/OpenCL).
\end{itemize}

\subsection{Hybrid Optical-Electronic AI Architectures}
\begin{itemize}
    \item \textbf{Co-Design Framework}: Integrate optical convolution blocks with electronic processors for nonlinear activation functions (ReLU, sigmoid, etc.).
    \item \textbf{Data Conversion Overheads}: Model the latency and power overhead of converting optical signals back to electrical domains for post-processing.
    \item \textbf{Algorithmic Adjustments}: Investigate techniques (e.g., binarization or low-precision networks) that mitigate noise and enhance robustness to optical imperfections.
\end{itemize}

\subsection{Simulation Tools and Validation}
\begin{itemize}
    \item \textbf{MATLAB/Python}: Primary tools for system-level simulation of 4f optics, WDM channels, and basic device modeling.
    \item \textbf{COMSOL or Lumerical (Optional)}: For high-fidelity electromagnetic (EM) simulation of wave propagation, lens aberrations, and waveguide components.
    \item \textbf{Deep Learning Frameworks}: PyTorch/TensorFlow for training and inference tasks, integrated with custom optical layers.
\end{itemize}

\section{Evaluation Metrics}
To quantify performance gains and viability, we will measure:
\begin{itemize}
    \item \textbf{Speedup Factor}: Ratio of optical convolution time to GPU/FPGA-based convolution time.
    \item \textbf{Energy Efficiency}: Estimated TeraFLOPS per watt, factoring in laser sources, modulators, and detection circuitry.
    \item \textbf{Scalability}: Throughput as the input size or the number of WDM channels increases.
    \item \textbf{Accuracy}: Deviation of optical computations (due to noise, quantization, or misalignment) from ideal electronic computations in representative AI tasks.
\end{itemize}

\section{Project Timeline}
Table~\ref{timeline} outlines the major phases over a six-month period.

\begin{table}[h]
\centering
\caption{Proposed Project Timeline}
\label{timeline}
\begin{tabular}{|c|p{5.5cm}|c|}
\hline
\textbf{Phase} & \textbf{Task} & \textbf{Duration} \\
\hline
1 & Literature Review and Preliminary Simulations & 4 weeks \\
2 & Fourier Optics Simulation (4f System) & 6 weeks \\
3 & WDM-Based Parallel Processing & 6 weeks \\
4 & Hybrid Optical-Electronic AI Model & 4 weeks \\
5 & Performance Analysis and Benchmarking & 4 weeks \\
6 & Research Report, Presentation, and Final Review & 4 weeks \\
\hline
\end{tabular}
\end{table}

\section{Budget Considerations}
Because this study is predominantly simulation-based, costs remain relatively modest. Table~\ref{budget} summarizes the anticipated expenditures.

\begin{table}[h]
\centering
\caption{Estimated Budget}
\label{budget}
\begin{tabular}{|l|c|}
\hline
\textbf{Resource} & \textbf{Estimated Cost} \\
\hline
MATLAB and COMSOL Licenses & Provided by University \\
High-Performance Computing Resources & University Cluster Access \\
Misc. (Printing, Conference Fees) & \$500 \\
\hline
\end{tabular}
\end{table}

\section{Expected Outcomes and Future Directions}
By the end of this project, we anticipate:
\begin{itemize}
    \item A validated simulation framework that integrates realistic optical components with AI computation flows.
    \item Quantitative evidence (speedup, energy consumption, accuracy metrics) that supports or challenges the viability of optical computing solutions.
    \item Design insights into the trade-offs between purely optical systems and hybrid optical-electronic approaches, highlighting optimal parameter selections (e.g., channel spacing in WDM, lens characteristics, or waveguide geometry).
    \item Recommendations for future hardware prototypes, detailing critical parameters (laser power, modulator sensitivity, waveguide coupling) necessary for practical photonic computing.
\end{itemize}

\section{Conclusion}
This proposal addresses the growing need for alternative computational paradigms by simulating and benchmarking optical computing architectures for AI and high-speed signal processing. By focusing on Fourier-based convolution, WDM parallelism, and hybrid co-design with electronic hardware, we aim to demonstrate the potential advantages of photonic approaches while acknowledging their inherent practical challenges. The outcomes of this study will help guide subsequent experimental implementations and advance the field of optical computing toward energy-efficient, high-throughput AI acceleration.

\section*{Acknowledgments}
The authors would like to thank the Drexel University Department of Electrical and Computer Engineering for providing the resources and support necessary to undertake this research.

\bibliographystyle{IEEEtran}
\begin{thebibliography}{00}
\bibitem{jouppi2017datacenter} N. P. Jouppi, \emph{et al.}, ``In-Datacenter Performance Analysis of a Tensor Processing Unit,'' \emph{Proceedings of the 44th Annual International Symposium on Computer Architecture (ISCA)}, 2017, pp. 1--12.

\bibitem{goodman2005introduction} J. W. Goodman, \emph{Introduction to Fourier Optics}. Roberts and Company, 2005.

\bibitem{shen2017deep} Y. Shen, \emph{et al.}, ``Deep Learning with Coherent Nanophotonic Circuits,'' \emph{Nature Photonics}, vol. 11, no. 7, 2017, pp. 441--446.

\bibitem{hughes2018training} T. W. Hughes, \emph{et al.}, ``Training of Photonic Neural Networks through in situ Backpropagation and Gradient Measurement,'' \emph{Optica}, vol. 5, no. 7, 2018, pp. 864--871.
\end{thebibliography}

\end{document}